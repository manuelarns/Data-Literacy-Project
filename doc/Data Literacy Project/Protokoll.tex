\documentclass[11pt]{article}

\usepackage{sectsty}
\usepackage{graphicx}

% Margins
\topmargin=-0.45in
\evensidemargin=0in
\oddsidemargin=0in
\textwidth=6.5in
\textheight=9.0in
\headsep=0.25in

\title{ Title}
\author{ Author }
\date{\today}

\begin{document}
\maketitle	
\pagebreak

% Optional TOC
% \tableofcontents
% \pagebreak

%--Paper--

\section{Protokoll}
Protokoll Rücksprache 27.01.:

Bewertungskritierien:
1.Ambitiousness of the project
2.Presentation (Plots etc
3.Text(Formulierung )
4.Code(Stackoverflow, Nachvollziehbar)

Ziel:
a)Rekonstruktion möglich
b) Ergebnisse sehen

Limit sind 4 Seiten (Appendix- Formulierung in Mail) Tutoren haben limitiert Zeit

Paper: Angenommen guten Kollegen Erzählen: Folgende Biases: Selection bias, wenig DAten, Leute können lügen
Data collection: 

Spearman ein, zwei Sätze FOkusentscheidung, ob man hier Seitenplatz verschenken möchte
Spearman Theorie muss nicht super detailliert rein

VL Sachen müssen nicht genauer erklärt werden 

Zielgruppe ist Student aus unserem Studium, öffne und mache es kurz nach, We use alpha gleich weil Cross validation so und so

Implementation erwähnen

Binärer Prediktor ist ok, wegen rank corelation, nochmal googeln

Regression: unkorrelierte Predictors implizieren nicht independence macht uns confident, dass linear unabhängig

Analyse sollte nicht zu post-hoc sein. Lieber beobachtend, erster Schritt ist beobachtend, am Ende Interpretation

Outlier detection bei Kasachstan
Alle statistischen Methoden Ad-hoc wegen zu wenig Daten 
Wir nehmen mindestens 3 samples um outliers zu verhinden

Regression auf Basis aller 5 aber wir schauen uns alle großen
Regressionsgütekriterium

Erwähnen, dass zwei Prediktoren nicht verwendet (

Psychologie: replication crisis —> 70 Prozent nicht geklappt —> Anreize in Akademia die Science verhindern

Font size ist wichtig in Grafiken!



\pagebreak
\section{Section 2}
Lorem Ipsum \\

%--/Paper--

\end{document}